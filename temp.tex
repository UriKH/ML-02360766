%\title{Hebrew document in WriteLatex - מסמך בעברית}
\documentclass{article}
\usepackage{amsmath, amssymb}
\usepackage{mathtools, nccmath}
\usepackage[utf8x]{inputenc}
\usepackage{centernot}          % for "not" on symbols
\usepackage{graphicx}
\usepackage{textcase}
\usepackage{subcaption} % load 'caption' automatically
\usepackage[english, hebrew]{babel}
\selectlanguage{hebrew}
\usepackage[top=2cm,bottom=2cm,left=2.5cm,right=2cm]{geometry}

\graphicspath{ {./images/} }

\usepackage{fixmath}    % for bold math symbols
\usepackage{siunitx}
\usepackage[explicit]{titlesec}

% Math utility and parentheses
\newcommand{\eqdef}{\mathrm{:=}}
\newcommand{\ceil}[1]{\left\lceil{#1}\right\rceil }     % ceil
\newcommand{\floor}[1]{\left\lfloor{#1}\right\rfloor}   % floor
\newcommand{\curly}[1]{\left\{{#1}\right\}}             % curly
\newcommand{\angular}[1]{\left\langle{#1}\right\rangle} % angle brackets
\newcommand{\rect}[1]{\left[{#1}\right]}                % rectangular
\newcommand{\p}[1]{\left({#1}\right)}                   % matching height brackets
\newcommand{\vll}[1]{\left|{#1}\right|}
\newcommand{\vl}[2]{\left.{#1}\right|{#2}}
\newcommand{\bb}[1]{\mathbb{#1}}                        % for number groups
\newcommand{\modulo}[3]{{#1}\equiv{#2}\;{\p{\mathrm{mod}\;{#3}}}} % modulo computation
\newcommand{\eps}{\varepsilon}
\newcommand{\fra}[2]{\p{\frac{#1}{#2}}}                 % fraction with parenthesis
\newcommand{\dl}[1]{\left. {{#1}} \right|}               % add long | delim

% complexity notation
\newcommand{\Oof}[1]{O\p{{#1}}}                     
\newcommand{\oof}[1]{o\p{{#1}}}
\newcommand{\tof}[1]{\Theta\p{{#1}}}
\newcommand{\Omof}[1]{\Omega\p{{#1}}}
\newcommand{\omof}[1]{\omega\p{{#1}}}

% probability notations
\newcommand{\cond}[2]{P\p{{#1}|{#2}}}       % conditional probability
\newcommand{\prob}[1]{P\p{{#1}}}            % probability notation
\newcommand{\bayes}[2]{\frac{P\p{{#2}|{#1}} \cdot P\p{#1}}{P\p{{#2}}}} % bayes formula
\newcommand{\E}[1]{\bb{E}\rect{{#1}}}       % expected value
\newcommand{\EE}[1]{\E{\E{#1}}}             % E[E[]]
\newcommand{\V}[1]{\mathrm{Var}\p{{#1}}}    % variance
\newcommand{\C}[1]{\mathrm{Cov}\p{{#1}}}    % covariance

% Formatting utility
\newcommand{\n}{\newline}                       % new line
\newcommand{\vtab}{\vspace{1em}\noindent}       % vertical space
\newcommand{\ds}{\textendash\;}                 % dash with space
\newcommand{\mds}{\textendash}                  % math dash without space
\newcommand{\question}[1]{\section*{שאלה {#1}}}
\newcommand{\nquestion}[1]{\newpage\section*{שאלה {#1}}}
\newcommand{\subsec}[1]{\subsection*{סעיף {#1}'}}
\newcommand{\smm}[2]{\sum_{#1}^{#2}}
\newcommand{\sm}[1]{\sum_{#1}}
\newcommand{\txt}[1]{\mathrm{#1}}
\newcommand{\bd}[1]{\mathbf{#1}}

% Qantum physics
\DeclarePairedDelimiter\bra{\langle}{\rvert}
\DeclarePairedDelimiter\ket{\lvert}{\rangle}
\DeclarePairedDelimiterX\bk[2]{\langle}{\rangle}{#1\,\delimsize\vert\,\mathopen{}#2}
\DeclarePairedDelimiterX\bkt[3]{\langle}{\rangle}{#1\,\delimsize\vert\,#2\,\delimsize\vert\,\mathopen{}#3}
\newcommand{\A}{\textup{\AA}}
\newcommand{\um}[1]{~\rect{\txt{#1}}}
\newcommand{\st}[2]{{#1}_{\txt{#2}}}
\newcommand{\e}[1]{\times 10^{#1}}
\newcommand{\op}[1]{\hat{#1}}       % hat symbol for operators
\newcommand{\ct}[1]{{#1}^\dag}      % complex transpose

% constants
\newcommand{\h}{6.626\e{-34}}       % Plank's constant Js
\newcommand{\rh}{1.054\e{-34}}      % reduced Plank's constant Js
\newcommand{\sol}{3\e{8}}           % speed of light m/s
\newcommand{\ec}{1.602\e{-19}}      % electron charge C
\newcommand{\ecom}{\e{-19}}         % electron charge order of magnitude
\newcommand{\emass}{9.109\e{-31}}   % electron mass
\newcommand{\centi}{\e{-2}}
\newcommand{\milli}{\e{-3}}
\newcommand{\micro}{\e{-6}}
\newcommand{\nano}{\e{-9}}
\newcommand{\expt}{\e{-10}}
\newcommand{\pico}{\e{-12}}

% Linear algebra
\usepackage{amsmath}
\usepackage{xparse}
\usepackage{expl3}
\ExplSyntaxOn
% one temporary token list
\tl_new:N \l__mymatrix_tl
\RenewDocumentCommand{\vec}{m}{ \begin{pmatrix} \clist_use:nn { #1 } { \\ } \end{pmatrix}}      % columns vector
\NewDocumentCommand{\rvec}{m}{ \begin{pmatrix} \clist_use:nn { #1 } { & } \end{pmatrix}}        % row vector
\NewDocumentCommand{\tvec}{m}{ \begin{pmatrix} \clist_use:nn { #1 } { \\ } \end{pmatrix}^\top}  % transposed columns vector
\NewDocumentCommand{\rtvec}{m}{ \begin{pmatrix} \clist_use:nn { #1 } { & } \end{pmatrix}^\top } % transposed row vector
% — Matrix via regex replacement —
\RenewDocumentCommand{\matrix}{m}{ \begin{pmatrix}
    \tl_set:Nn \l__mymatrix_tl { #1 }                               % 1. load input into a token list
    \regex_replace_all:nnN { \s*;\s* } { \c{\\} } \l__mymatrix_tl   % 2. normalize: remove spaces around ; and ,
    \regex_replace_all:nnN { \s*,\s* } { & } \l__mymatrix_tl
    \tl_use:N \l__mymatrix_tl                                       % 3. typeset the result
\end{pmatrix}}
\ExplSyntaxOff


\begin{document}


\begingroup
    \centering
    \LARGE פיזיקה קוונטית להנדסה\\
    \Large תרגיל בית יבש 2 \\[1 em]
    \large \today\\[0.5em]
    \large אורי כשר חיטין \par
    \large \L{215105321} \par
\endgroup
\rule{\textwidth}{0.4pt}

\section*{אורך גל דה ברויי ועקרון אי הוודאות}
\subsection*{סעיף א'}
במהלך חישוב אורך גל דה ברויי, על מנת לברר האם צריך להשתמש בנוסחאות המכניקה הקלאסית או בנוסחאות תורת היחסות הפרטית עלינו לברר האם מתקיים: $E_{\text{\L{rest}}} \gg E_k$ \ds שכן ראינו כי תנאי זה מתקיים עבור מהירויות לא יחסותיות.\n

\vtab
בהתאם לאינווריאנט לורנץ האנרגיה הכוללת של החלקיק הינה:
\[
    E^2=p^2c^2+m^2c^4 \Rightarrow E = \sqrt{p^2c^2+m^2c^4}
\]
מהצד השני: 
\[
    E_{\text{\L{total}}}=E_0+E_k \Rightarrow E_k = E - E_0
\]
כמו כן, אנרגיית המנוחה של חלקיק הינה: $E_0 = mc^2$\n
לפיכך נקבל:
\[
    E_k = \sqrt{p^2c^2+m^2c^4} - mc^2 = \sqrt{p^2c^2+\p{mc^2}^2} - mc^2
\]
נרצה לבטל את חלקו של $mc^2$ בביטוי האנרגיה ולקבל את הקירוב הקלאסי:
\[
    E_k = \sqrt{p^2c^2+\p{mc^2}^2} - mc^2 = mc^2\sqrt{\frac{p^2}{m^2c^2}+1}-mc^2
\]
נשים לב כי עבור הקירוב $\frac{p^2}{m^2c^2} \ll 1$ נוכל לבצע קירוב טיילור מסדר ראשון לשורש:
\begin{gather*}
    f(x)=\sqrt{1+x},~ f(0) = 1, ~f'(0)=\frac{1}{2} \Rightarrow f\p{\frac{p^2}{m^2c^2}} = 1+\frac{1}{2}\frac{p^2}{m^2c^2}
\end{gather*}
עבור הקירוב לעיל נקבל את הקירוב הקלאסי:
\[
    E_k \approx mc^2\p{1+\frac{1}{2}\frac{p^2}{m^2c^2}}-mc^2=\frac{p^2}{2m}
\]
אם מתקיים התנאי נוכל להשתמש באנרגיה שמצאנו על מנת לברר את קיום התנאי שהצגנו עבור שימוש בנוחסאות הקלאסיות:
\[
    \frac{1}{2}\frac{p^2}{m^2c^2} = \frac{\frac{p^2}{2m}}{mc^2} = \frac{E_k}{E_0} \ll 1 \Rightarrow E_k \ll E_0
\]

\subsection*{סעיף ב'}
נתון: $\Delta E_k = 0.01 E_k, ~E_k=10\um{eV}$\n

\vtab
ראשית נבדוק האם תנועת האלקטרון דורשת שימוש במשוואות יחסות פרטית.
נשים לב כי האנגריה הקינטית של האלקטרון קטנה מאוד ביחס לאנרגיית המנוחה שלו: $E_0 = 8.1981\e{-14}\um{J}=0.512 \um{MeV}$\n
מכאן נסיק כי תנועת האלקטרון אינה יחסותית. \n

\vtab
לפי עקרון אי הוודאות במיקום ובתנע מתקיים: 
\[
    \Delta p\Delta x \ge \frac{\hbar}{2} \Rightarrow \Delta x \ge \frac{\hbar}{2\Delta p}
\]
נקבל מינימום עבור $\Delta x$ כאשר יתקיים שוויון:
\[
    \Delta x_{\min} = \frac{\hbar}{2\Delta p}
\]
הקשר בין התנע והאנרגיה הקינטית נתון על ידי: $E_k=\frac{p^2}{2m}$\n
נשתמש בקשר זה ונקבל ונמצא את הקשר בין אי הוודאות במיקום לאי הוודאות באנרגיה:
\begin{gather*}
    p = \sqrt{2mE_k} \\
    dp = \sqrt{2m}\frac{dE_k}{2\sqrt{E_k}} = \sqrt{\frac{m}{2E_k}}dE_k \Rightarrow \Delta p = \sqrt{\frac{m}{2E_k}}\Delta E_k
    \\
    \Delta x \ge \frac{\hbar}{2\Delta p} = \frac{\hbar}{2\sqrt{\frac{m}{2E_k}} \Delta E_k} = \frac{\hbar}{\Delta E_k}\sqrt{\frac{E_k}{2m}}
\end{gather*}
נציב:
\[
    \Delta x_{\min} = \frac{\hbar}{\Delta E_k}\sqrt{\frac{E_k}{2m}} = \frac{\rh}{0.01 \times (10 \times \ec)}\sqrt{\frac{10 \times \ec}{2\times \emass}}=6.169 \e{-9} = 61.69 \um{\A}
\]

\subsection*{סעיף ג'}
כעת האלקטרון בעל אנגריה קינטית $E_K = 5\um{MeV}$ ולכן כעת לא מתקיים $E_k \ll E_0$. פירוש הדבר הוא שכעת עלינו לבצע חישוב בעזרת משוואות היחסות הפרטית.
\n

\vtab
כעת בשונה מהסעיף הקודם נשתמש בקשר בין אנרגיה ותנע הנתון על ידי אינווריאנט לורנץ:
\begin{align*}
    E^2 &= \p{E_0 + E_k}^2 = p^2c^2 + \p{mc^2}^2 \\ 
    &\Rightarrow p^2c^2 = (E_0+E_k)^2 + E_0^2 \\
    & \Rightarrow p = \frac{1}{c}\sqrt{(E_0+E_k)^2 + E_0^2}
\end{align*}
נמצא את קשר אי הוודאות בין תנע ואנרגיה:
\begin{gather*}
    dp = \frac{1}{2c}\p{\p{E_0+E_k}^2 + E_0^2}^{-\frac{1}{2}}\p{2E_k + 2E_0}dE_k = \frac{E_k+E_0}{c\sqrt{(E_0+E_k)^2 + E_0^2}}dE_k \\
    \Delta p = \frac{E_k+E_0}{c\sqrt{(E_0+E_k)^2 + E_0^2}}\Delta E_k \\
    \Delta x \ge \frac{\hbar}{2\Delta p} = \frac{\hbar c \sqrt{(E_0+E_k)^2 + E_0^2}}{2\p{E_k+E_0}\Delta E_k}
\end{gather*}
נציב: 
\[
     \Delta x_{\min} =  \frac{\hbar c \sqrt{(E_0+E_k)^2 + E_0^2}}{2\p{E_k+E_0}\Delta E_k} = 
     \frac{\rh \times \sol \times \sqrt{\p{\emass \times \p{\sol}^2 + 10 \times \ec }^2 + \p{\emass \times \p{\sol}^2}}}{2\p{\emass \times \p{\sol}^2 + 10 \times \ec }\times 0.01 \times \p{10 \times \ec} } = 
     0.0165 \um[\A]
\]

\subsection*{סעיף ד'}
נתון חלקיק לא יחסותי בעל מסה המבצע תנועה סביב $X=0$ כאשר האנרגיה הפוטנציאלית שלו נתונה לפי: $U=\frac{1}{2}m\omega^2 x^2, ~\omega = \sqrt{\frac{k}{m}}$\n

\vtab
מכאן שהאנרגיה הכוללת של החלקיק הינה:
\[
    E = \underbrace{\frac{p^2}{2m}}_{E_k} + \underbrace{\frac{1}{2}m\omega^2 x^2}_{U}
\]
אנו מחפשים $\angular{E}$ אשר נתונה לפי המשוואה:
\[
    \angular{E} = \angular{\frac{p^2}{2m}} + \angular{\frac{1}{2}m\omega^2 x^2} = \frac{\angular{p^2}}{2m} + \frac{1}{2}m\omega^2 \angular{x^2}
\]
נכונות המעבר נובעת מכך שפרט לגדלים $E,~x,~p$ יתר האיברים קבועים.\n

\vtab
נזכור כי ברצוננו למצוא ביטוי עבור ממוצע האנרגיה כפונקציה של אי הוודאות של $x$. לפיכך נשתמש בקשרים המוכרים עבור אי הוודאות:
\begin{gather*}
    \p{\Delta x}^2 = \angular{x^2}-\angular{x}^2 \\
    \p{\Delta p}^2 = \angular{p^2}-\angular{p}^2 \\
\end{gather*}
מכיוון שהתנועת החלקיק היא תנועה פשוטה סביב נקודה מסויימת, בפרט $x=0$, נקבל כי $\angular{x}=0, ~\angular{p}=0$.
לכן:
\[
    \p{\Delta x}^2 = \angular{x^2} \qquad \p{\Delta p}^2 = \angular{p^2}
\]

כעת מעקרון אי הוודאות עבור התנע והמיקום:
\[
    \Delta x \Delta p \ge \frac{\hbar}{2} \Rightarrow \Delta p \ge \frac{\hbar}{2\Delta x}
\]
נציב את כל הביטויים שמצאנו בביטוי האנרגיה הממוצעת ונקבל:
\begin{align*}
    \angular{E} &= \frac{\angular{p^2}}{2m} + \frac{1}{2}m\omega^2 \angular{x^2} = \frac{\p{\Delta p}^2}{2m} + \frac{1}{2}m\omega^2 \p{\Delta x}^2 \\
    & = \frac{\p{\Delta p}^2}{2m} + \frac{1}{2}m\omega^2 \p{\Delta x}^2 \ge \frac{\fra{\hbar}{2\Delta x}^2}{2m} + \frac{1}{2}m\omega^2 \p{\Delta x}^2 =
    \boxed{\frac{\hbar^2}{8m\p{\Delta x}^2} + \frac{1}{2}m\omega^2 \p{\Delta x}^2}
\end{align*}
נשים לב כי אם נחשב עבור אי וודאות מינימלית בתנע נקבל שוויון.

\subsection*{סעיף ה'}
נחפש את האנרגיה הממוצעת המינימלית של החלקיק. נעשה זאת על ידי גזירת פונקציית האנרגיה וחישוב נקודת המינימום. \n

\vtab
\begin{gather*}
    \frac{d\angular{E}}{d\p{\Delta x}} = -\frac{\hbar^2}{4m\p{\Delta x}^3} + m\omega^2 \Delta x \\
    \frac{d\angular{E}}{d\p{\Delta x}} = 0 \Rightarrow m\omega^2 \Delta x = \frac{\hbar^2}{4m\p{\Delta x}^3} 
    \Rightarrow \p{\Delta x}^4 = \frac{\hbar^2}{2m^2\omega^2} \Rightarrow \Delta x = \sqrt{\frac{\hbar}{2m\omega}}
\end{gather*}
מצאנו את ערך ה\ds$\Delta x$ היחיד עבורו הנגזרת מתאפסת. זהו גם הערך המתאים עבור האנרגיה הממוצעת המינימאלית.\n
נציב ערך זה ב\ds$\angular{E}\p{\Delta x}$ ונקבל את $\angular{E}_{\min}$.
\[
    \angular{E} = \boxed{\frac{\hbar^2}{8m\p{\sqrt{\frac{\hbar}{2m\omega}}}^2} + \frac{1}{2}m\omega^2 \p{\sqrt{\frac{\hbar}{2m\omega}}}^2} 
    = \frac{\hbar \omega}{4} + \frac{\hbar \omega}{4} = \boxed{\frac{\hbar \omega}{2}}
\]

\section{הולכה חשמלית ואורך גל דה ברויי}
\subsection*{סעיף א'}
נתונה התפלגות מהירויות חלקיקים חופשיים לא יחסותיים: 
\[
    f(v)d^3v = \frac{m}{2\pi k_B T}^\frac{3}{2} \cdot e^{-\frac{mv^2}{2k_BT}}d^3v
\]
ראשית נמצא את $f(v)$ בעזרת הרמז ונוסחאת העזר:
\begin{gather*}
    f(v) d^3v = f(v)\cdot v^2 \sin{\theta} dv d\theta d\Phi \\
    \Downarrow\\
    \int_0^{2\pi}\int_0^{\pi} f(v)v^2 \sin{\theta} d\theta d\Phi = \int_0^{2\pi} d\Phi \int_{0}^{\pi} \sin{\theta}d\theta = 2\pi \cdot 2 = 4\pi
    \\
    f(v)dv = 4\pi v^2 = 4\pi v^2 \frac{m}{2\pi k_B T}^\frac{3}{2} \cdot e^{-\frac{mv^2}{2k_BT}}dv
    \\
\end{gather*}

נשים לב כי התנע הממוצע מקיים: $\angular{p} = \angular{mv} = m \angular{v}$\n
נחשב את המהירות הממוצעת בהתאם להתפלגות $f(v)$:
\begin{align*}
    \int_0^\infty vf(v)dv = \int_0^\infty 4\pi 4\pi v^3 \frac{m}{2\pi k_B T}^\frac{3}{2} \cdot e^{-\frac{mv^2}{2k_BT}}dv =
    4\pi \p{\frac{m}{2\pi k_B T}}^\frac{3}{2} \int_0^\infty v^3 e^{-\p{\frac{m}{2k_B T}}v^2}dv \underbrace{=}_{\text{\L{hint}}} \frac{1}{2\p{\frac{m}{2k_B T}}^2}
\end{align*}

\end{document}